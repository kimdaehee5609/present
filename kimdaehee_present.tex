
%	-------------------------------------------------------------------------------
%
%			작성		
%				2020년 
%				10월 
%				26일 
%				월
%
%
%
%
%
%
%	-------------------------------------------------------------------------------

%	\documentclass[10pt,xcolor=pdftex,dvipsnames,table]{beamer}
%	\documentclass[10pt,blue,xcolor=pdftex,dvipsnames,table,handout]{beamer}
%	\documentclass[14pt,blue,xcolor=pdftex,dvipsnames,table,handout]{beamer}
	\documentclass[aspectratio=1610,20pt,xcolor=pdftex,dvipsnames,table,handout]{beamer}
%	\documentclass[aspectratio=169,17pt,xcolor=pdftex,dvipsnames,table,handout]{beamer}
%	\documentclass[aspectratio=149,17pt,xcolor=pdftex,dvipsnames,table,handout]{beamer}
%	\documentclass[aspectratio=54,17pt,xcolor=pdftex,dvipsnames,table,handout]{beamer}
%	\documentclass[aspectratio=43,17pt,xcolor=pdftex,dvipsnames,table,handout]{beamer}
%	\documentclass[aspectratio=32,17pt,xcolor=pdftex,dvipsnames,table,handout]{beamer}

		% Font Size
		%	default font size : 11 pt
		%	8,9,10,11,12,14,17,20
		%
		% 	put frame titles 
		% 		1) 	slideatop
		%		2) 	slide centered
		%
		%	navigation bar
		% 		1)	compress
		%		2)	uncompressed
		%
		%	Color
		%		1) blue
		%		2) red
		%		3) brown
		%		4) black and white	
		%
		%	Output
		%		1)  	[default]	
		%		2)	[handout]		for PDF handouts
		%		3) 	[trans]		for PDF transparency
		%		4)	[notes=hide/show/only]

		%	Text and Math Font
		% 		1)	[sans]
		% 		2)	[sefif]
		%		3) 	[mathsans]
		%		4)	[mathserif]


		%	---------------------------------------------------------	
		%	슬라이드 크기 설정 ( 128mm X 96mm )
		%	---------------------------------------------------------	
%			\setbeamersize{text margin left=2mm}
%			\setbeamersize{text margin right=2mm}

		%	---------------------------------------------------------	
		%	슬라이드 크기 설정 ( 128mm X 96mm )
		%	---------------------------------------------------------	

%			% Format presentation size to A4
%			\usepackage[size=a4]{beamerposter}		% A4용지 크기 사용
			\geometry{paper=a5paper}
%			% Format presentation size to A4 길게
%			\geometry{paper=a4paper, landscape}

			\setbeamersize{text margin left=10mm}
			\setbeamersize{text margin right=10mm}


	%	========================================================== 	Package
		\usepackage{kotex}						% 한글 사용
		\usepackage{amssymb,amsfonts,amsmath}	% 수학 수식 사용
		\usepackage{color}					%
		\usepackage{colortbl}					%

		\usepackage{caption}		% 테이블 사용 건 2020.10.22
		\usepackage{tabularx} 	% 테이블 사용 건 2020.10.22


	%		========================================================= 	note 옵션인 
	%			\setbeameroption{show only notes}
		

	%		========================================================= 	Theme

		%	---------------------------------------------------------	
		%	전체 테마
		%	---------------------------------------------------------	
		%	테마 명명의 관례 : 도시 이름
%			\usetheme{default}			%
%			\usetheme{Madrid}    		%
%			\usetheme{CambridgeUS}    	% -red, no navigation bar
%			\usetheme{Antibes}			% -blueish, tree-like navigation bar

		%	----------------- table of contents in sidebar
			\usetheme{Berkeley}		% -blueish, table of contents in sidebar
									% 개인적으로 마음에 듬

%			\usetheme{Marburg}			% - sidebar on the right
%			\usetheme{Hannover}		% 왼쪽에 마크
%			\usetheme{Berlin}			% - navigation bar in the headline
%			\usetheme{Szeged}			% - navigation bar in the headline, horizontal lines
%			\usetheme{Malmoe}			% - section/subsection in the headline

%			\usetheme{Singapore}
%			\usetheme{Amsterdam}

		%	---------------------------------------------------------	
		%	색 테마
		%	---------------------------------------------------------	
%			\usecolortheme{albatross}	% 바탕 파란
%			\usecolortheme{crane}		% 바탕 흰색
%			\usecolortheme{beetle}		% 바탕 회색
%			\usecolortheme{dove}		% 전체적으로 흰색
%			\usecolortheme{fly}		% 전체적으로 회색
%			\usecolortheme{seagull}	% 휜색
%			\usecolortheme{wolverine}	& 제목이 노란색
%			\usecolortheme{beaver}

		%	---------------------------------------------------------	
		%	Inner Color Theme 			내부 색 테마 ( 블록의 색 )
		%	---------------------------------------------------------	

%			\usecolortheme{rose}		% 흰색
%			\usecolortheme{lily}		% 색 안 칠한다
%			\usecolortheme{orchid} 	% 진하게

		%	---------------------------------------------------------	
		%	Outter Color Theme 		외부 색 테마 ( 머리말, 고리말, 사이드바 )
		%	---------------------------------------------------------	

%			\usecolortheme{whale}		% 진하다
%			\usecolortheme{dolphin}	% 중간
%			\usecolortheme{seahorse}	% 연하다

		%	---------------------------------------------------------	
		%	Font Theme 				폰트 테마
		%	---------------------------------------------------------	
%			\usfonttheme{default}		
			\usefonttheme{serif}			
%			\usefonttheme{structurebold}			
%			\usefonttheme{structureitalicserif}			
%			\usefonttheme{structuresmallcapsserif}			



		%	---------------------------------------------------------	
		%	Inner Theme 				
		%	---------------------------------------------------------	

%			\useinnertheme{default}
			\useinnertheme{circles}		% 원문자			
%			\useinnertheme{rectangles}		% 사각문자			
%			\useinnertheme{rounded}			% 깨어짐
%			\useinnertheme{inmargin}			




		%	---------------------------------------------------------	
		%	이동 단추 삭제
		%	---------------------------------------------------------	
%			\setbeamertemplate{navigation symbols}{}

		%	---------------------------------------------------------	
		%	문서 정보 표시 꼬리말 적용
		%	---------------------------------------------------------	
%			\useoutertheme{infolines}


			
	%	---------------------------------------------------------- 	배경이미지 지정
%			\pgfdeclareimage[width=\paperwidth,height=\paperheight]{bgimage}{./fig/Chrysanthemum.jpg}
%			\setbeamertemplate{background canvas}{\pgfuseimage{bgimage}}

		%	---------------------------------------------------------	
		% 	본문 글꼴색 지정
		%	---------------------------------------------------------	
%			\setbeamercolor{normal text}{fg=purple}
%			\setbeamercolor{normal text}{fg=red!80}	% 숫자는 투명도 표시


		%	---------------------------------------------------------	
		%	itemize 모양 설정
		%	---------------------------------------------------------	
%			\setbeamertemplate{items}[ball]
%			\setbeamertemplate{items}[circle]
%			\setbeamertemplate{items}[rectangle]






		\setbeamercovered{dynamic}





		% --------------------------------- 	문서 기본 사항 설정
		\setcounter{secnumdepth}{3} 		% 문단 번호 깊이
		\setcounter{tocdepth}{3} 			% 문단 번호 깊이




% ------------------------------------------------------------------------------
% Begin document (Content goes below)
% ------------------------------------------------------------------------------
	\begin{document}
	

			\title{ 현황 정리 }
			\author{ 김대희 }
			\date{ 
					2020년 
					12월 
					26일
					토  
					} 


% -----------------------------------------------------------------------------
%		개정 내용
% -----------------------------------------------------------------------------
%
%		2020년 10월 7일 첫제작
%		2020년 10월 12일 서영 엔지니어링 추가
%		2020년 10월 15일 문수선원 추가 
%		2020년 10월 21일 반야선원 추가 
%		2020년 10월 26일 수채화 추가
%		2020년 10월 28일 위생 추가
%		2020년 11월 02일 청파팀 계좌번호 추가 
%		2020년 12월 13일 지오알엔디 추가
%		2020년 12월 16일 보현명상산악회 추가
%		2020년 12월 19일 위생 수정
%		2020년 12월 21일 월 	USB 정리
%		2020년 12월 25일 금	설치 프로그램
%		2020년 12월 26일 토	물품 구입
%
%
%
%


	%	==========================================================
	%
	%	---------------------------------------------------------- 1	page

		\begin{frame}[plain]
		\titlepage
		\end{frame}


	%	---------------------------------------------------------- 2	page
		\begin{frame} [plain]{목차}
		\tableofcontents%

			\setlength{\leftmargini}{ 2em}			
			\begin{itemize}

				\item [part1] \ref{part1}	차 시간 정리
				\item [part2] \ref{part2}	연락처
				\item [part3] \ref{part3}	기타현황
				\item [part4] \ref{part4}	좋은 문구
%				\item [part5] \ref{part5}	출입문			
%				\item [part6] \ref{part6}	일반사이트		
%				\item [part7] \ref{part7}	쇼핑 사이트		

			\end{itemize}



		\end{frame}

	%	---------------------------------------------------------- 3	page
		\begin{frame} [t,plain]
		\end{frame}						
	%	---------------------------------------------------------- 4	page
		\begin{frame} [t,plain]
		\end{frame}						



	%	========================================================== 조직
		\part{차 시간 정리 }
		\frame{\partpage}

\label{part1} 	%  차 시간 정리


		\begin{frame} [plain]{목차}
		\tableofcontents%
		\end{frame}
		
		\begin{frame} [plain]
		\end{frame}

	%	---------------------------------------------------------- 출퇴근
	%		Frame
	%	----------------------------------------------------------
		\section{출퇴근}


		\begin{frame} [t,plain]
		\frametitle{출퇴근}

			\begin{block} {출근}
			\setlength{\leftmargini}{2em}			
			\begin{itemize}
				\item 
			\end{itemize}
			\end{block}						

		   \begin{center}
%			\captionof{table}{Test}
			\label{table:second}
			\setlength{\tabcolsep}{2pt}
			\small
		     	\begin{tabular}{ c |c|c|c|c|c} \hline
				번호  	& 초량 		& 	부전  	& 교대 		& 오시 		&비고  \\ 
					  	& 	 		& 		  	& 	 		& 리아 		&	  \\ \hline  \hline
				1  		& 	05:28	&	5:30		&	5:37		&	6:00		&  \\ \hline
				2  		& 	05:44	&	5:50		&	5:57		&	6:20		&  \\ \hline
				3  		& 	05:58	&	6:10		&	6:17		&	6:40		&  \\ \hline
				4  		& 	06:10	&	6:25		&	6:32		&	6:55		&  \\ \hline
				5  		& 	06:22	&	6:39		&	6:46		&	7:09		&  \\ \hline
				6  		& 	06:34	&	6:54		&	7:01		&	7:24		&  \\ \hline
				7  		& 	06:46	&	7:09		&	7:16		&	7:39		&  \\ \hline
				8  		& 	06:58	&	7:27		&	7:34		&	7:57		&  \\ \hline
				9  		& 	07:09	&	7:48		&	7:55		&	8:18		&  \\ \hline
				10  		& 			&			&	8:05		&	8:30		&  \\ \hline
				11  		& 			&			&	8:20		&	8:43		&  \\ \hline
				12  		& 			&			&	8:35		&	8:58		&  \\ \hline
			\end{tabular}
			\end{center}%



		\end{frame}						


		\begin{frame} [t,plain]
		\frametitle{퇴근}

			\begin{block} {퇴근}
			\setlength{\leftmargini}{2em}			
			\begin{itemize}
				\item 
			\end{itemize}
			\end{block}						

		   \begin{center}
%			\captionof{table}{Test}
			\label{table:second}
			\setlength{\tabcolsep}{2pt}
			\small
		     	\begin{tabular}{ c |c|c|c|c|c|c} \hline
				번호  	& 오시		& 	교 대  	& 거 제 			& 거제 		&부전	& 비고 \\ 
					  	& 리아		& 	  		&  				& 해맞이 	&		\\ \hline \hline
																						
				1  		& 	18:12	&	18:40	&	18:42		&	18:44		&  \\ \hline
				2  		& 	18:38	&	19:02	&	19:03		&	19:05		&  \\ \hline
				3  		& 	19:02	&	19:26	&	19:27		&	19:29		&  \\ \hline
				4  		& 	19:17	&	19:41	&	19:42		&	19:44		&  \\ \hline
				5  		& 	19:30	&	19:57	&	19:59		&	20:01		&  \\ \hline
				6  		& 	19:50	&	20:14	&	20:15		&	20:17		&  \\ \hline
				7  		& 	20:04	&	20:28	&	20:29		&	20:31		&  \\ \hline
				8  		& 	20:21	&	20:45	&	20:46		&	21:48		&  \\ \hline
				9  		& 	20:39	&	21:03	&	21:04		&	21:06		&  \\ \hline
				10  		& 	20:58	&	21:22	&	21:23		&	21:25		&  \\ \hline
			\end{tabular}
			\end{center}%

				7:08 교대역 \hrulefill \\

				7:27 초량역 \hrulefill \\
				17분 소요 \hrulefill \\



		\end{frame}						


		
	%	---------------------------------------------------------- 철도
	%		Frame
	%	----------------------------------------------------------
		\section{철도}


	%	----------------------------------------------------------
		\begin{frame} [t,plain]
		\frametitle{철도}

			\begin{block} {철도 : 부전 방향 }
			\setlength{\leftmargini}{2em}			
			\begin{itemize}
				\item 
			\end{itemize}
			\end{block}						

		   \begin{center}
%			\captionof{table}{Test}
			\label{table:second}
			\setlength{\tabcolsep}{2pt}

%			\small {
			\footnotesize {
%			\tiny {

		     	\begin{tabularx}{\textwidth}{ 	m{0.12\textwidth} 	%1
										m{0.12\textwidth} 	%2
										m{0.12\textwidth}  	%3
										m{0.12\textwidth}  	%4
										m{0.12\textwidth}  	%5
										m{0.12\textwidth}  	%6
										m{0.12\textwidth}  	%7
										m{0.12\textwidth}    	%8
										}\hline

%		     	\begin{tabular}{ c |c|c|c|c|c} \hline
				번호  	& 남창		& 신  		& 센텀 			& 부전 		&비고	\\ 
					  	& 			& 해운대	  	&  				& 	 		&		\\ \hline \hline

					  	& 			& 	35분		& 7분 			& 10분	 	& 52분		\\ \hline \hline
																						
				1761	& 	07:31	&	08:06	&	08:13		&	08:23		&  \\ \hline
				1773	& 	09:39	&	10:21	&				&	10:35		&  \\ \hline
				1775	& 	10:41	&	11:18	&	11:26		&	11:36		&  \\ \hline

				1777	& 			&	12:14	&				&	11:36		&  \\ \hline

				1779	& 	12:45	&	13:21	&	13:29		&	13:39		&  \\ \hline
				1781	& 	13:49	&	14:23	&				&	14:38		&  \\ \hline

				1621	& 	14:26	&	15:03	&	15:10		&	15:20		&  \\ \hline
				1783	& 	16:05	&	16:44	&				&	16:59		&  \\ \hline

				1943	& 	\textbf{17:48}	&	18:24	&	18:32		&	18:44			&  \\ \hline
				1787	& 	19:05	&	19:42	&	19:49		&	19:59		&  \\ \hline
				1681	& 	22:34	&	23:12	&	23:19		&	23:29		&  \\ \hline

				1791	& 	21:30	&	22:09	&	22:17		&	22:27		&  \\ \hline
				1793	& 	22:14	&	22:47	&				&	23:02		&  \\ \hline
				1681	& 	22:34	&			&	23:19		&	23:39		&  \\ \hline
				1795	& 	23:37	&	00:13	&	00:20		&	00:30		&  \\ \hline
			\end{tabularx}
			}			
			\end{center}%


		\end{frame}						


	%	----------------------------------------------------------
		\begin{frame} [t,plain]
		\frametitle{철도}

			\begin{block} {철도 : 남창 방향 }
			\setlength{\leftmargini}{2em}			
			\begin{itemize}
				\item 
			\end{itemize}
			\end{block}						

		   \begin{center}
%			\captionof{table}{Test}
			\label{table:second}
			\setlength{\tabcolsep}{2pt}

%			\small {
			\footnotesize {
%			\tiny {

%		     	\begin{tabular}{ c |c|c|c|c|c} \hline
%		     	\begin{tabularx}{\textwidth}{ 	p{0.12\textwidth} 	%1
%										p{0.12\textwidth} 	%2
%										p{0.12\textwidth}  	%3
%										p{0.12\textwidth}  	%4
%										p{0.12\textwidth}  	%5
%										p{0.12\textwidth}  	%6
%										p{0.12\textwidth}  	%7
%										p{0.12\textwidth}    	%8
%										}\hline

		     	\begin{tabularx}{\textwidth}{ 	m{0.12\textwidth} 	%1
										m{0.12\textwidth} 	%2
										m{0.12\textwidth}  	%3
										m{0.12\textwidth}  	%4
										m{0.12\textwidth}  	%5
										m{0.12\textwidth}  	%6
										m{0.12\textwidth}  	%7
										m{0.12\textwidth}    	%8
										}\hline

				번호  	& 부전		& 센텀  		& 신			& 기장	& 좌천		& 남창	 		&비고	\\ 
					  	& 			&			& 해운대	  	&		&			& 				&			\\ \hline \hline
																						
					  	& 			&11			& 8	 	 	&11		&10			&13 				&			\\ 
					  	& 			&11			& 19	 	 	&30		&40			&53 				&			\\ \hline \hline

				1774	&	6:30		&	6:14	&	6:22	&	6:33	&	6:43	&	6:56	&	\\ \hline					
				\textbf{1622}	&	7:23		&	7:34	&	7:41	&	7:52	&	8:02	&	8:15	&	\\ \hline					
				1776	&	7:45		&	7:56	&	8:03	&	8:14	&	8:24	&	8:40	&	\\ \hline					
				1682	&	9:10		&	9:21	&	9:28	&	9:39	&	9:49	&	10:04	&	\\ \hline					
				1944	&	9:46		&	9:57	&	10:04	&	10:15	&	10:25	&	10:42	&	\\ \hline					
				1780	&	11:33	&	11:45	&	11:53	&	12:03	&	12:13	&	12:26	&	\\ \hline					
				1782	&	13:00	&	-	&	13:16	&	13:27	&	-	&	-	&	\\ \hline					
				1784	&	13:54	&	14:05	&	14:12	&	14:23	&	14:34	&	14:50	&	\\ \hline					
				1888	&	15:35	&	15:46	&	15:53	&	16:04	&	16:15	&	16:29	&	\\ \hline					
				1790	&	16:37	&	-	&	16:53	&	17:04	&	17:15	&	-	&	\\ \hline					
				1792	&	17:35	&	17:46	&	17:54	&	18:05	&	18:16	&	18:32	&	\\ \hline					
				1762	&	19:05	&	19:16	&	19:23	&	19:34	&	19:44	&	20:00	&	\\ \hline					
				1794	&	20:14	&	20:25	&	20:33	&	20:43	&	20:53	&	21:07	&	\\ \hline					
				1796	&	21:03	&	-	&	21:19	&	21:29	&	21:40	&	21:55	&	\\ \hline					
			\end{tabularx}
			}
			\end{center}%


		\end{frame}						

	%	----------------------------------------------------------
		\begin{frame} [t,plain]
		\frametitle{철도}

			\begin{block} {철도}
			\setlength{\leftmargini}{2em}			
			\begin{itemize}
				\item  	무궁화호 1776호 
				\item 	부전 7:45 출발
				\item 	남창 8:40 도착
				\item 	남창 택시 052) 238 - 3129 
			\end{itemize}
			\end{block}						


			\begin{block} {동해 남부선 복선 전철}
			\setlength{\leftmargini}{2em}			
			\begin{itemize}
				\item  	
				\item 	
				\item 	
			\end{itemize}
			\end{block}						


		\end{frame}						

	%	----------------------------------------------------------
		\begin{frame} [t,plain]
		\end{frame}						


	%	---------------------------------------------------------- 출근시간표}
	%		Frame
	%	----------------------------------------------------------
		\section{출근시간표}
		\begin{frame} [t,plain]
		\frametitle{출근시간표}
			\begin{block} {출근시간표}
			\setlength{\leftmargini}{2em}			
			\begin{itemize}
				\item 
				\item 
				\item 
			\end{itemize}
			\end{block}				

		   \begin{center}
%			\captionof{table}{출근시간표}
			\label{table:second}
			\setlength{\tabcolsep}{2pt}
			\small
		     	\begin{tabular}{ c |c|c|c|c|c} \hline
				번호  	& 초량		& 교대  		&교대		& 오시리아 		&비고	\\ 
					  	& 			&			& 		  	&  				&			\\ \hline \hline
																						
				1		& 	6:22		&	6:41		&	6:46		&	7:09		&  \\ \hline
				2		& 	6:34		&	6:53		&	7:01		&	7:24		&  \\ \hline
				3		& 	6:46		&	7:05		&	7:16		&	7:39		&  \\ \hline
				4		& 	6:58		&	7:17		&	7:34		&	7:57		&  \\ \hline

			\end{tabular}
			\end{center}%

				집  			\hrulefill 7:40 \hrulefill출발 \\
				초량역 		\hrulefill 7:49 \hrulefill출발 \\
				교대역		\hrulefill 8:08 \hrulefill도착 \\
				교대역 		\hrulefill 8:20 \hrulefill출발 \\
				오시리아역 	\hrulefill 8:43 \hrulefill도착 \\
				동부산TG	\hrulefill 9:00 \hrulefill출발 \\

		
		\end{frame}						






	%	---------------------------------------------------------- 퇴근 시간표}
	%		Frame
	%	----------------------------------------------------------
		\section{퇴근 시간표}
		\begin{frame} [t,plain]
		\frametitle{}
			\begin{block} {퇴근 시간표}
			\setlength{\leftmargini}{2em}			
			\begin{itemize}
				\item  	평일\\
						18:38	 오시리아 출발 \\
						19:02	 교대 도착  \\
						19:08	 교대 출발  \\
						19:27	 초량 도착 19분  \\
				\item 	15:03 \\
						15:31 \\
						16:02 \\
						16:29 \\

				\item  	금요일\\
						16:29	 오시리아 출발 \\
						16:57	 교대 도착  28분 소요 \\
						17:04	 교대 출발  \\
						17:23	 초량 도착  \\
				\item 	17:01 \\
						17:32 \\
						17:54 \\

			\end{itemize}
			\end{block}				


		\end{frame}						

	%	---------------------------------------------------------- 버스 시간표}
	%		Frame
	%	----------------------------------------------------------
		\section{버스 시간표}
		\begin{frame} [t,plain]
		\frametitle{버스 시간표}

%			\begin{block} {버스 시간표}
%			\setlength{\leftmargini}{2em}			
%			\end{block}						

			\begin{block} {내리 16058 당사입구방면 }
			\setlength{\leftmargini}{1em}			
			\begin{itemize}
				\item \textbf{38} 	\hfill 	청강리 - 황령터널 - 민주공원
				\item 39  			\hfill 	교리 - 부경대 - 용호동
				\item \textbf{40} 	\hfill 	청강리 - 경성대 - 구덕운동장
				\item 63 			\hfill 	청강리 - 양정 - 진구청
				\item 182 			\hfill 	장산역 - 일광 - 정관
				\item 185 			\hfill 	장산역 - 송정 - 청강리 
				\item 200			\hfill 	청강리 - 동래역 - 북구청
			\end{itemize}
			\end{block}						

			\begin{block} {해운대 반야 선원}
			\setlength{\leftmargini}{1em}			
			\begin{itemize}
				\item 좌동재래시장입구 		\hfill	185
				\item 좌동재래시장입구 		\hfill	182
				\item 장산역 해운대 문화회관 	\hfill	38
				\item 부흥등학교 			\hfill	 39
				\item 삼정코아아파트 			\hfill 	200
				\item 부흥고,해운대백병원 	\hfill 	40
			\end{itemize}
			\end{block}						


		\end{frame}						

	%	---------------------------------------------------------- 지하철 시간표}
	%		Frame
	%	----------------------------------------------------------
		\section{지하철 시간표}
		\begin{frame} [t,plain]
		\frametitle{}

			\begin{block} {지하철 시간표}
			\setlength{\leftmargini}{2em}			
			\begin{itemize}
				\item  장산 - 서면 : 18개역 34분
				\item  서면 - 초량 : 5개역 9분
				\item 
			\end{itemize}
			\end{block}						


		   \begin{center}
%			\captionof{table}{출근시간표}
			\label{table:second}
			\setlength{\tabcolsep}{0pt}


%			\small {
			\footnotesize {
%			\tiny {

%		     	\begin{tabularx}{\textwidth}{ l | l | r |r  }\hline
		     	\begin{tabularx}{\textwidth}{ 		b{0.17\textwidth} 
											b{0.17\textwidth} 
											b{0.17\textwidth} 
											b{0.17\textwidth} 
											b{0.17\textwidth} 
											b{0.15\textwidth} 
									}\hline

				번호   	& 장산역	& 서면  		&서면 		& 초량	&비고			\\ \hline				\hline				
					   	& 		& 34분 		&		 	& 9분	&				\\ \hline				\hline				

				1   		& 20:00	& 20:33  		&	20:44 	& 20:54		&			\\ \hline
				2   		& 20:06	& 20:39  		&	20:52 	& 21:02		&			\\ \hline
				3   		& 20:12	& 20:45  		&	21:00 	& 21:10		&			\\ \hline
				4   		& 20:18	& 20:51  		&	21:08 	& 21:18		&			\\ \hline
				5   		& 20:24	& 20:57  		&	21:16 	& 21:26		&			\\ \hline
			\end{tabularx}
			}
			\end{center}%					


			\begin{block} {지하철 시간표}
			\setlength{\leftmargini}{2em}			
			\begin{itemize}
				\item  7:08 교대역
				\item  7:27 초량 도착
				\item  17분 소요 
			\end{itemize}
			\end{block}						


		\end{frame}						


	%	---------------------------------------------------------- 차량 시간표}
	%		Frame
	%	----------------------------------------------------------
		\section{차량 시간표}
		\begin{frame} [t,plain]
		\frametitle{차량 시간표}
			\begin{block} {차량 시간표}
			\setlength{\leftmargini}{2em}			
			\begin{itemize}
				\item 24.1 km
				\item  38분
				\item 온양IC
			\end{itemize}
			\end{block}						
		\end{frame}						

	%	---------------------------------------------------------- 3	page
		\begin{frame} [t,plain]
		\end{frame}						
	%	---------------------------------------------------------- 4	page
		\begin{frame} [t,plain]
		\end{frame}						


	%	========================================================== 연락처
	%	
	%	
	%	
	%	---------------------------------------------------------- 1	page
		\part{연락처}
		\frame{\partpage}

\label{part2} 	% 연락처

		
	%	---------------------------------------------------------- 2	page
		\begin{frame} [plain]{목차}
		\tableofcontents%
		\end{frame}


	%	---------------------------------------------------------- 3	page
		\begin{frame} [t,plain]
		\end{frame}						
	%	---------------------------------------------------------- 4	page
		\begin{frame} [t,plain]
		\end{frame}						


	%	---------------------------------------------------------- 청출
	%		Frame
	%	---------------------------------------------------------- 	page	1
		\section{청출 051)504-5658}
		\begin{frame} [t,plain]
		\frametitle{청출}
			\begin{block} {청출 051)504-5658}
			\setlength{\leftmargini}{1em}			
			\begin{itemize}
				\item 박일
				\item 오동재
				\item 심경환
				\item 강민재
				\item 김민자
				\item 홍성화
				\item 송윤석
				\item 이상홍
				\item 
			\end{itemize}
			\end{block}			

								
		\end{frame}						
	

	%	---------------------------------------------------------- 지오알엔디
	%		Frame
	%	---------------------------------------------------------- 	page	2
		\section{지오알엔디 }
		\begin{frame} [t,plain]
		\frametitle{지오알엔디}
			\begin{block} {지오알엔디 051) 515 - 0138 }
			\setlength{\leftmargini}{1em}			
			\begin{itemize}
				\item 박이근
				\item 김향은 부장
				\item 퍼스트 인셈텀 906호
				\item 2호선 센텀시티역 하차 4번 출구 도로5분
				\item 용송정밀 \\한림면 안하리 1110-3번지
				\item 버스 : 181 송정2단지 주공 승차 SK텔레콤 하차
				\item 
			\end{itemize}
			\end{block}			
		\end{frame}						

	%	---------------------------------------------------------- 	page	3
		\begin{frame} [t,plain]
		\end{frame}						
	%	---------------------------------------------------------- 	page	4
		\begin{frame} [t,plain]
		\end{frame}						

	%	---------------------------------------------------------- 초량 집
	%		Frame
	%	---------------------------------------------------------- 	page	1
		\section{초량 집}

		\begin{frame} [t,plain]
		\frametitle{초량 집 - 주소 }
			\begin{block} {초량 집 - 주소 }
			\setlength{\leftmargini}{1em}			
			\begin{itemize}
				\item 	48798
				\item 	부산시동구 중앙대로 251번길 28 (초량동)
				\item 	W3W \hrulefill ///시선, 필요한, 색칠 
			\end{itemize}
			\end{block}			
		\end{frame}						
								

	%	---------------------------------------------------------- 	page	2
		\begin{frame} [t,plain]
		\frametitle{초량 집 - 구성원 }
			\begin{block} {초량 집 - 구성원}
			\setlength{\leftmargini}{1em}			
			\begin{itemize}
				\item 	김 대희 \hrulefill		   3839-5609 \\
						65 09 24 을사 (뱀) 57

				\item 	안 신영 \hrulefill		   9919-5609  \\
						69 08 16  기유 (닭) 53

				\item 	김 재민 \hrulefill		   2871-5609 \\
						96 05 12 병자 (쥐) 26

				\item 	김 재경 \hrulefill		   9805-5609 \\
						98 05 21 무인 (호랑이) 24

				\item 	김 재홍 \hrulefill		   9988-5609 \\
						02 11 02 임오 (말) 20

				\item 	조 경연 \hrulefill		   8917-5619 \\
						41 01 04 신사 (뱀) 81

				\item 	전 삼연 \hrulefill		    6654-1384 \\
			\end{itemize}
			\end{block}			
		\end{frame}						


	%	---------------------------------------------------------- 	page	3
		\begin{frame} [t,plain]
		\frametitle{초량 집}
			\begin{block} {녹산}
			\setlength{\leftmargini}{1em}			
			\begin{itemize}
				\item 송정동 서삼종 01  6777 1282 

			\end{itemize}
			\end{block}			

		\end{frame}						


	%	---------------------------------------------------------- 	page	4
		\begin{frame} [t,plain]
		\end{frame}						


	%	---------------------------------------------------------- 울산지사 }
	%		Frame
	%	----------------------------------------------------------
		\section{울산지사 }

	%	---------------------------------------------------------- 1 page
		\begin{frame} [t,plain]
		\frametitle{울산지사 }

			\begin{block} {울산지사 : 비번 3366 }
			\setlength{\leftmargini}{2em}			
			\begin{itemize}
				\item 44971
				\item 울산광역시 울주군 온양읍 광정로 796-66 \\
			\end{itemize}

			\begin{itemize}
				\item ARS \hrulefill 052) 701 -  6200 
				\item 상황실  \hrulefill 052) 701 -  6271
				\item 상황실  \hrulefill 052) 701 -  6272
			\end{itemize}

			\begin{itemize}
				\item fax 고객지원팀 \hrulefill 052) 701 -  6297
				\item fax 도로안전팀 \hrulefill 052) 701 -  6298
				\item fax 상황실 \hrulefill 052) 701 -  6299
			\end{itemize}

			\begin{itemize}
				\item 지사장 \hrulefill 박태완
				\item 고객지원팀장 \hrulefill 윤남구
				\item 도로안전팀장 \hrulefill 손정줄
				\item 교통안전팀장 \hrulefill 하규영
			\end{itemize}
			\end{block}						





		\end{frame}						

	%	---------------------------------------------------------- 2 page
		\begin{frame} [t,plain]
		\frametitle{울산지사 : 고객지원팀 }

			\begin{block} {울산지사 : 고객지원팀 }
			\setlength{\leftmargini}{2em}			
			\begin{itemize}
				\item 운남규 팀장 	\hrulefill  8806 5700
				\item 이충열 관리차장 \hrulefill 3547 3867
				\item 남기환	영업차장	\hrulefill 3738 0270
				\item 
				\item 
			\end{itemize}
			\end{block}						

		\end{frame}						

	%	---------------------------------------------------------- 3 page
		\begin{frame} [t,plain]
		\frametitle{울산지사 : 도로안전팀 }

			\begin{block} {울산지사 : 도로안전팀 }
			\setlength{\leftmargini}{2em}			
%			\begin{itemize}
%				\item 
%				\item 
%			\end{itemize}
			\end{block}	

		   \begin{center}
%			\captionof{table}{출근시간표}
			\label{table:second}
			\setlength{\tabcolsep}{0pt}

%			\small {
			\footnotesize {
%			\tiny {

%		     	\begin{tabularx}{\textwidth}{ l | l | r |r  }\hline
		     	\begin{tabularx}{\textwidth}{ p{0.18\textwidth} p{0.22\textwidth} p{0.4\textwidth} p{0.2\textwidth}   }\hline

				성명   	& 직책		& 전번  		&비고			\\ \hline				\hline				
				손정줄	&	팀장	&	010 - 6312 - 8506 	&		  \\ \hline				
				조재성	&	차장	&	010 - 7745 - 9290	&		 \\ \hline				
				장호진	&	차장대우	&	010 - 8541 - 0167	&	시설물	 \\ \hline				\hline				
				한재승	&	과장	&	010 - 4847 - 0785	&	설비	 \\ \hline				
				박자민	&	과장	&	010 - 3098 - 0015	&	품질	 \\ \hline				
				이영주	&	과장 대우	&	010 - 2004 - 5076	&	전기	 \\ \hline				\hline				
				김병우	&	대리	&	010 - 9084 - 2299	&	상황	 \\ \hline				
				고아현	&	대리	&	010 - 3371 - 5914	&	조경	 \\ \hline				
				이효찬	&	대리	&	010 - 5349 - 5347	&	구조물	 \\ \hline				\hline				
				이병근	&	대리	&	010 - 4895 - 8985	&	공구장	 \\ \hline				
				권영일	&	대리	&	010 - 4160 - 0042	&	공구장	 \\ \hline				\hline				
				김태건	&	선임	&	010 - 9808 - 8653	&	구조물	 \\ \hline				
				김형일	&	선임	&	010 - 3077 - 4984	&		 \\ \hline				
				한우덕	&	선임	&	010 - 3737 - 9096	&		 \\ \hline				
				손동영	&	선임	&	010 - 3144 - 7119	&	구조물	 \\ \hline				\hline				
				권동우	&	사원	&	010 - 8735 - 1397	&		 \\ \hline				
				한승훈	&	사원	&	010 - 4746 - 4434	&		 \\ \hline				
				이상희	&	사원	&	010 - 2399 - 9635	&		 \\ \hline				
				임기식	&	사원	&	010 - 3434 - 9635	&		 \\ \hline				
				이원빈	&	사원	&		&		 \\ \hline				\hline				

			\end{tabularx}
			}
			\end{center}%					



		\end{frame}						


	%	---------------------------------------------------------- 4 page
		\begin{frame} [t,plain]
		\frametitle{울산지사 : 교통안전팀 }

			\begin{block} {울산지사 : 교통안전팀 }
			\setlength{\leftmargini}{2em}			
			\begin{itemize}
				\item 
				\item 
				\item 
			\end{itemize}
			\end{block}						

		   \begin{center}
%			\captionof{table}{교통안전팀}
			\label{table:second}
			\setlength{\tabcolsep}{0pt}

%			\small {
%			\footnotesize {
			\tiny {
		     	\begin{tabularx}{\textwidth}{ p{0.18\textwidth} p{0.22\textwidth} p{0.4\textwidth} p{0.2\textwidth}   }\hline
				성명   	& 직책		& 전번  		&비고			\\ \hline				\hline				

				하규영	&	팀장			&	010 - 9145 - 5246	&		 \\ \hline		
				김중모	&	과장			&	010 - 4772 - 7719	&	교통 장비	 \\ \hline		
				최주학	&	과장			&	010 - 9397 - 3817	&	정보통신	 \\ \hline		
				정경진	&	과장대우		&	010 - 3567 - 2776	&		 \\ \hline	\hline		

				박성철	&	대리	&		&	운전	 \\ \hline		
				류정훈	&	대리	&		&	운전	 \\ \hline		
				박종철	&	대리	&		&		 \\ \hline		
				김회용	&	대리	&		&		 \\ \hline		
				김성인	&	대리	&		&	교통상황	 \\ \hline  \hline		
				엄성섭	&	선임	&		&		 \\ \hline		
				김승규	&	선임	&		&		 \\ \hline		
				김중기	&	선임	&		&		 \\ \hline		
				안상율	&	주임	&		&		 \\ \hline	  \hline			
													
				하점규	&	사원	&		&	주관상황관리	 \\ \hline		
				김학권	&	사원	&		&	순찰직	 \\ \hline		
				박성호	&	사원	&		&		 \\ \hline		
				김인석	&	사원	&		&	안전순찰	 \\ \hline		
				문정환	&	사원	&		&		 \\ \hline		
				김홍태	&	사원	&		&	안전순찰	 \\ \hline		
				이태양	&	사원	&		&		 \\ \hline		
				김성재	&	사원	&		&	안전순찰	 \\ \hline		
				박주렬	&	사원	&		&	안전순찰	 \\ \hline		
				최재덕	&	사원	&		&	안전순찰	 \\ \hline		
				최황원	&	사원	&		&	안전순찰	 \\ \hline		
				진명훈	&	사원	&		&	안전순찰	 \\ \hline		
				류용철	&	사원	&		&		 \\ \hline		
				유병열	&	사원	&		&	안전순찰	 \\ \hline		
				조경성	&	사원	&		&	안전순찰	 \\ \hline		
				김경민	&	사원	&		&	안전순찰	 \\ \hline		
				방우섭	&	사원	&		&	안전순찰	 \\ \hline		
				박영근	&	사원	&		&	안전순찰	 \\ \hline		


			\end{tabularx}
}
			\end{center}%					




		\end{frame}						

	%	---------------------------------------------------------- 능인회 }
	%		Frame
	%	----------------------------------------------------------  page	1
		\section{능인회 }
		\begin{frame} [t,plain]
		\frametitle{능인회 }

			\begin{block} {능인회 }
			\setlength{\leftmargini}{2em}			
			\begin{itemize}
				\item 월 : 정일현 : 01월 \hrulefill
				\item 화 : 여종한 : \hrulefill 11월
				\item 수 : 조승희 : \hrulefill 02월
				\item 목 : 조현진 : \hrulefill 03월
				\item 금 : 전대성 : \hrulefill 12월
				\item 양도근 : \hrulefill 10월

			\end{itemize}
			\end{block}	

		\end{frame}						


	%	---------------------------------------------------------- 반야선원}
	%		Frame
	%	----------------------------------------------------------  page	2
		\section{반야선원 }
		\begin{frame} [t,plain]
		\frametitle{반야선원 }

			\begin{block} {반야선원}

			\setlength{\leftmargini}{2em}			
			\begin{itemize}
				\item 48106 \hrulefill
				\item 부산광역시 해운대구 세실로87
				\item 영진파스타8층 \hrulefill
				\item 반야선원 \hrulefill
				\item 허공 스님 \hrulefill
				\item 010 - 3732 - 3383 \hrulefill
				\item 효담 스님 \hrulefill
				\item 010 - 4215 - 9119 \hrulefill

			\end{itemize}
			\end{block}						
				

			\begin{block} {반야선원 : 비밀번호 }

			\setlength{\leftmargini}{2em}			
			\begin{itemize}
				\item 9004*
				\item 9009*
				\item 23680 출입문 번호키
			\end{itemize}
			\end{block}						

		\end{frame}						


	%	---------------------------------------------------------- 문수선원
	%		Frame
	%	----------------------------------------------------------  page	3
		\section{문수선원 }
		\begin{frame} [t,plain]
		\frametitle{ }
			\begin{block} {문수선원 : 여천 무비 스님 }
			\setlength{\leftmargini}{1em}			
			\begin{itemize}
				\item \hrulefill	문수 경전 연구회 		
						\\ 첫째주 월요일 오후 3시-6시
				\item \hrulefill	금요법회 				
						\\	금요일 10시 30분
						\\ 1주 화엄경 강좌
						\\ 2주 참선
						\\ 3주 화엄경 강좌
						\\ 4주 법화경 사경
				\item \hrulefill	문수불교대학원			
						\\ 수요일 10시, 오후 7시
				\item \hrulefill	문수 사경반				
						\\ 금요일 오후 1시
				\item \hrulefill 친불회 					
						\\ 월요일 오전 10시
				\item \hrulefill 염화실
			\end{itemize}
			\end{block}						


		\end{frame}						



	%	---------------------------------------------------------- 여래선원
	%		Frame
	%	----------------------------------------------------------  page	4
		\section{여래선원 }
		\begin{frame} [t,plain]
		\frametitle{ }
			\begin{block} {여래선원 : 효산 스님 }
			\setlength{\leftmargini}{1em}			
			\begin{itemize}
				\item 여여선원 선원장 정여 스님
				\item 
				\item 
				\item 
			\end{itemize}
			\end{block}						


		\end{frame}						

	%	---------------------------------------------------------- 붓다선원
	%		Frame
	%	----------------------------------------------------------  page	5
		\section{붓다선원 }
		\begin{frame} [t,plain]
		\frametitle{ }
			\begin{block} {붓다선원 : 도경  스님 }
			\setlength{\leftmargini}{1em}			
			\begin{itemize}
				\item 문자 010 6257 1150
				\item 
				\item 
				\item 
			\end{itemize}
			\end{block}						

		\end{frame}						

	%	----------------------------------------------------------  page	6
		\begin{frame} [t,plain]
		\end{frame}						
	%	----------------------------------------------------------  page	7
		\begin{frame} [t,plain]
		\end{frame}						
	%	----------------------------------------------------------  page	8
		\begin{frame} [t,plain]
		\end{frame}						
			

	%	---------------------------------------------------------- 포교사단 }
	%		Frame
	%	----------------------------------------------------------
		\section{포교사단 }

	%	----------------------------------------------------------  page	1
		\begin{frame} [t,plain]
		\frametitle{포교사단 }

			\begin{block} {포교사단  : 051) 633-3011 }
			\setlength{\leftmargini}{2em}			
			\begin{itemize}
				\item 사단장 :  자재천 정분남
				\item 사무국장 :  손삼호  \\ jeamho son353535
				\item 간사 : 이원희 간사 010 9338 4658
				\item 교육부장 :  법인 김병인 \\ 
			\end{itemize}
			\end{block}						


		\end{frame}						


	%	----------------------------------------------------------  page	2
		\begin{frame} [t,plain]
		\frametitle{포교사단 }

			\begin{block} {포교사단  : 051) 633-3011 }
			\setlength{\leftmargini}{2em}			
			\begin{itemize}
				\item 중부지역
				\item 중동부 지역
				\item 동부 지역
				\item 서부 지역
				\item 남부 지역
				\item 지역중앙
				\item NGO사단본부
			\end{itemize}
			\end{block}						
		\end{frame}						


	%	----------------------------------------------------------  page		3
		\begin{frame} [t,plain]
		\frametitle{포교사단 }
			\begin{block} {포교사증  : A3-64-0048}
			\setlength{\leftmargini}{2em}			
			\begin{itemize}
				\item 자격번호 : A3-64-0048
				\item 성명 : 김대희(보조)
				\item 생년월일 : 1965.09.24.

				\item 자격취득일 : 2020.10.08.
				\item 갱신일 : 2025.10.07.
			\end{itemize}
			\end{block}						

		\end{frame}						

	%	---------------------------------------------------------- 	page	4
		\begin{frame} [t,plain]
		\end{frame}						

	%	---------------------------------------------------------- 포교사단 청파팀 }
	%		Frame
	%	---------------------------------------------------------- 	page	5
		\section{포교사단팀 청파팀 }
		\begin{frame} [t,plain]
		\frametitle{포교사단팀 청파팀 }
			\begin{block} {포교사단팀 청파팀 (군포교)}
			\setlength{\leftmargini}{2em}			
			\begin{itemize}
				\item 서부지역 총괄팀 
				\item 18 김영식 010 5311 7744
				\item 010 8949 3468
				\item 17 월인 김필근 
				\item 도선 고영옥 \\2020.12.15 공인중개사 합격
			\end{itemize}
			\end{block}						
		\end{frame}						

	%	---------------------------------------------------------- 포교사단 청파팀 경덕회 }
	%		Frame
	%	---------------------------------------------------------- 	page	6
		\begin{frame} [t,plain]
			\begin{block} {포교사단팀 청파팀 경덕회 }
			\setlength{\leftmargini}{2em}			
			\begin{itemize}
				\item 3명
				\item 매월 1인당 50,000원 보시
				\item 구경옥
				\item 김혜영
				\item 이연교
			\end{itemize}
			\end{block}						
		\end{frame}						


	%	---------------------------------------------------------- 포교사단 청파팀 법은회 }
	%		Frame
	%	---------------------------------------------------------- 	page	7
		\begin{frame} [t,plain]
			\begin{block} {포교사단팀 청파팀 법은회 }
			\setlength{\leftmargini}{2em}			
			\begin{itemize}
				\item 매월 첫주에 봉사하는 지원 단체
				\item 
				\item 
				\item 
				\item 
				\item 
				\item 
				\item 
			\end{itemize}
			\end{block}						
		\end{frame}						

	%	---------------------------------------------------------- 	page	8
		\begin{frame} [t,plain]
		\end{frame}						


	%	---------------------------------------------------------- 범포회}
	%		Frame
	%	---------------------------------------------------------- 	page	1
		\section{범포회}
		\begin{frame} [t,plain]
		\frametitle{범포회}
			\begin{block} {범포회}
			\setlength{\leftmargini}{1em}			
			\begin{itemize}
				\item 14기 경행 최우영 (48표) 금정팀
				\item 18기 염제 강창식 (39표) 동산팀
				\item 19기 상명심 김현이
				\item 20기 법인 김병인
			\end{itemize}
			\end{block}						
		\end{frame}						



	%	---------------------------------------------------------- 금정불교대학}
	%		Frame
	%	---------------------------------------------------------- 	
		\section{금정불교대학}

	%	---------------------------------------------------------- 	page	2
		\begin{frame} [t,plain]
		\frametitle{금정불교대학}

			\begin{block} {금정불교대학}
			\setlength{\leftmargini}{2em}			
			\begin{itemize}
				\item 
				\item 
				\item 
			\end{itemize}
			\end{block}						


			\begin{block} {여래사 불교대학}
			\setlength{\leftmargini}{2em}			
			\begin{itemize}
				\item 학장 원범스님
				\item 
				\item 
			\end{itemize}
			\end{block}						

		\end{frame}						

	%	---------------------------------------------------------- 금정불교대학 동문회
	%		Frame
	%	---------------------------------------------------------- 	page	3
		\begin{frame} [t,plain]
		\frametitle{금정불교대학 동문회}

			\begin{block} {금정불교대학}
			\setlength{\leftmargini}{2em}			
			\begin{itemize}
				\item 46219
				\item 부산 금정구 남산동 팔송로7번길5 \\ 범불대 총동문회 \\(남산동, 범어사불교대학)
				\item 부산광역시 금정구 남산동21-1 (범어사불교대학)
			\end{itemize}
			\end{block}						

			\begin{block} {창화 서종현}
			\setlength{\leftmargini}{2em}			
			\begin{itemize}
				\item
				\item
				\item
			\end{itemize}
			\end{block}						


			\begin{block} {신기열}
			\setlength{\leftmargini}{2em}			
			\begin{itemize}
				\item 부산시 생활체육회 이사
				\item 목원개발
				\item 대영기획
			\end{itemize}
			\end{block}						

		\end{frame}						

	%	---------------------------------------------------------- 	page	4
		\begin{frame} [t,plain]
		\end{frame}						



	%	---------------------------------------------------------- 보현 명상 산악회
	%		Frame
	%	----------------------------------------------------------
		\section{보현 명상 산악회}

	%	---------------------------------------------------------- page	1
		\begin{frame} [t,plain]
		\frametitle{보현 명상 산악회}

			\begin{block} {보현 명상 산악회}
			\setlength{\leftmargini}{2em}			
			\begin{itemize}
				\item 조회진 보살
				\item 정성희 보살
				\item 이수경 보살
				\item 류상영 법사
				\item 김영수 명상맨
				\item 박민 회장
				\item 
				\item 이상미
			\end{itemize}
			\end{block}						
		\end{frame}						

	%	----------------------------------------------------------  page		2 	효심사
		\begin{frame} [t,plain]
		\frametitle{효심사}

			\begin{block} {효심사}
			\setlength{\leftmargini}{2em}			
			\begin{itemize}
				\item 효문 스님
				\item 
				\item 
			\end{itemize}
			\end{block}						
		\end{frame}						

	%	---------------------------------------------------------- 3 page	불교 문화재 해설사
		\begin{frame} [t,plain]
		\frametitle{불교 문화재 해설사}

			\begin{block} {불교 문화재 해설사}
			\setlength{\leftmargini}{2em}			
			\begin{itemize}
				\item 
				\item 
				\item 
			\end{itemize}
			\end{block}						
		\end{frame}						

	%	---------------------------------------------------------- 2 page	대불청
		\begin{frame} [t,plain]
		\frametitle{대불청}

			\begin{block} {대불청}
			\setlength{\leftmargini}{2em}			
			\begin{itemize}
				\item 
				\item 
				\item 
			\end{itemize}
			\end{block}						
		\end{frame}						




	%	---------------------------------------------------------- 금성고등학교
	%		Frame
	%	---------------------------------------------------------- 	page	1
		\section{금성고등학교}
		\begin{frame} [t,plain]
		\frametitle{금성고등학교}
			\begin{block} {금성고등학교 총동창회 }
			\setlength{\leftmargini}{2em}			
			\begin{itemize}
				\item 	총동창회 회장  \hrulefill 박정태 \\
						22대,25회	\\
						010 - 3551 - 3806
				\item 총동창회 회장  \hrulefill 박경호
				\item 총동창회 총무 \hrulefill 김광택
				\item 28기 회장	\hrulefill 	김재환
				\item 

			\end{itemize}
			\end{block}						
		\end{frame}						

	%	---------------------------------------------------------- 금성고등학교 총동창회
	%		Frame
	%	---------------------------------------------------------- 	page	2
%		\section{금성고등학교 총동창회}
		\begin{frame} [t,plain]
		\frametitle{ }
			\begin{block} {금성고등학교 총동창회 }
			\setlength{\leftmargini}{2em}			
			\begin{itemize}
				\item 금산회\hrulefill
				\item 금성 하모니\hrulefill
				\item 금남회
				\item 금진회
				\item 금우회
				\item 샛별회
				\item 연동회
				\item 동해남부회
				\item 비단재회
				\item 세우회
				\item 장학재단



			\end{itemize}
			\end{block}						
		\end{frame}						

	%	---------------------------------------------------------- 금성고등학교 28회
	%		Frame
	%	---------------------------------------------------------- 	page	3
%		\section{금성고등학교 28회 }
		\begin{frame} [t,plain]
		\frametitle{ }
			\begin{block} {금성고등학교 28회 - 임원}
			\setlength{\leftmargini}{2em}			
			\begin{itemize}
				\item 28기 회장	\hrulefill 	김 재환
				\item 28기 총무	\hrulefill 	김 대희
			\end{itemize}
			\end{block}					

			\begin{block} {금성고등학교 28회 - 반대표}
			\setlength{\leftmargini}{2em}			
			\begin{itemize}
				\item 01반	\hrulefill 	
				\item 02반	\hrulefill 	
				\item 03반	\hrulefill 	
				\item 04반	\hrulefill 	
				\item 05반	\hrulefill 	
				\item 06반	\hrulefill 	
				\item 07반	\hrulefill 	
				\item 08반	\hrulefill 	
				\item 09반	\hrulefill 	
				\item 10반	\hrulefill 	
			\end{itemize}
			\end{block}						
	
		\end{frame}						


	%	---------------------------------------------------------- 금성고등학교 28회 3학년4반
	%		Frame
	%	---------------------------------------------------------- 	page	3
		\begin{frame} [t,plain]
		\frametitle{ }
			\begin{block} {금성고등학교 28회 - 3학년 4반}
			\setlength{\leftmargini}{2em}			
			\begin{itemize}
				\item 신인재	\hrulefill 	
				\item 이승민	\hrulefill 	
				\item 주보선	\hrulefill 	
				\item 강은철	\hrulefill 	
				\item 김대곤	\hrulefill 	
				\item 김면	\hrulefill 	
				\item 김태형	\hrulefill 	
				\item 김현강	\hrulefill 	
				\item 김현중	\hrulefill 	
				\item 이상백	\hrulefill 	
				\item 문중현	\hrulefill 	
				\item 서윤진	\hrulefill 	
				\item 양천을	\hrulefill 	
				\item 오장근	\hrulefill 	
				\item 조찬래	\hrulefill 	
				\item 최기환	\hrulefill 	
				\item 한영만	\hrulefill 	
			\end{itemize}
			\end{block}						
	
		\end{frame}						

	%	---------------------------------------------------------- 서영엔지니어링
	%		Frame
	%	---------------------------------------------------------- 	page	1
		\section{서영엔지니어링 }
		\begin{frame} [t,plain]
		\frametitle{ }
			\begin{block} {서영엔지니어링 }
			\setlength{\leftmargini}{2em}			
			\begin{itemize}
				\item 김경갑
				\item 이재봉 : 와룡법전
				\item 하수진
				\item 정경진 전무


			\end{itemize}
			\end{block}						
		\end{frame}						

	%	---------------------------------------------------------- 	page	2
		\begin{frame} [t,plain]
		\end{frame}						

	%	---------------------------------------------------------- 택배사 }
	%		Frame
	%	---------------------------------------------------------- 	page	3
		\section{택배사 }
		\begin{frame} [t,plain]
		\frametitle{택배사 }
			\begin{block} {택배사 }
			\setlength{\leftmargini}{2em}			
			\begin{itemize}
				\item 로젠택배 	\hrulefill 1588 - 9988 \\
						함준호 	\hrulefill 	010 - 3221 - 8874 	
						\\ 	\hrulefill 	
						
				\item CJ대한통운 \hrulefill \\
						김범한 	\hrulefill	010 - 3114 - 5481 \\
								\hrulefill 	010 - 7913 - 3694 
						\\ 	\hrulefill 	

				\item 롯데 택배 	\hrulefill 	1588 - 2121 \\
						김병홍	\hrulefill	010 - 5607 - 3811 	
						\\ 	\hrulefill 	

				\item 한진 택배	\hrulefill 
								\hrulefill 010 - 2579 - 5668	
						\\ 	\hrulefill 	

				\item 합동 택배	\hrulefill 
								\hrulefill 
						\\ 	\hrulefill 	

				\item 우체국		\hrulefill
						\\ 	\hrulefill 	

				\item 
			\end{itemize}
			\end{block}						
		\end{frame}						

	%	---------------------------------------------------------- 	page	4
		\begin{frame} [t,plain]
		\end{frame}						


	%	========================================================== 기타
	%
	%	---------------------------------------------------------- 1	page

		\part{기타 현황}
		\frame{\partpage}
		
\label{part3} 	%  키타 현황

	%	---------------------------------------------------------- 2	page
		\begin{frame} [plain]{목차}
		\tableofcontents%
		\end{frame}

	%	---------------------------------------------------------- 3	page
		\begin{frame} [t,plain]
		\end{frame}						
	%	---------------------------------------------------------- 4	page
		\begin{frame} [t,plain]
		\end{frame}						

	%	---------------------------------------------------------- 계좌번호 }
	%		Frame
	%	---------------------------------------------------------- page	1
		\section{계좌번호 }

		\begin{frame} [t,plain]
		\frametitle{계좌번호 }
			\begin{block} {계좌번호 : 식구 }
			\setlength{\leftmargini}{1em}			
			\begin{itemize}
				\item 	김대희	\hrulefill \\  						카뱅	\hrulefill 3333 - 04 -711 66 92  \\
																	신한	\hrulefill 371-02-190034 \\
																	기업은행	\hrulefill 			\\
																	우체국	\hrulefill 		\\
									

				\item 	안신영	\hrulefill  \\ 						농협		\hrulefill 821049 52 128362
				\item 	김재민	\hrulefill \\ 						부산		\hrulefill 10 120 2477 5001
				\item 	김재경	\hrulefill \\ 						카뱅		\hrulefill 3333 - 16 - 589 42 20
				\item 	김재홍	\hrulefill \\ 						국민		\hrulefill 9465 - 0101 - 430754
			\end{itemize}
			\end{block}						
		\end{frame}						

	%	---------------------------------------------------------- page	2
		\begin{frame} [t,plain]
			\begin{block} {계좌번호: 업체  }
			\setlength{\leftmargini}{1em}			
			\begin{itemize}
				\item 	오동재	\hrulefill \\ 						하나				\hrulefill 1869 - 1018 - 556607
				\item 	최정임 	(머리)	\hrulefill \\  				부산				\hrulefill 082020 - 196 - 765 
				\item 	장은경 	(식대)	\hrulefill \\ 				경남				\hrulefill 6982 - 10065621
				\item 	이동인 	(풍물)	\hrulefill \\ 				기업				\hrulefill 0104 - 144 - 1572

				\item 	자운사 청파팀		\hrulefill \\ 				우체국			\hrulefill 601567-01-004343

				\item 
			\end{itemize}
			\end{block}						
		\end{frame}						


	%	---------------------------------------------------------- page	3
		\begin{frame} [t,plain]
		\frametitle{계좌 이체 - 카뱅}
			\begin{block} {계좌 이체 - 카뱅  }
			\setlength{\leftmargini}{1em}			
			\begin{itemize}
				\item 	적금				\hrulefill 5일 \\ 						20,000			\hrulefill
				\item 	포교사단			\hrulefill CMS \\ 									\hrulefill
				\item 	자운사 청파팀		\hrulefill 20일 \\ 						20,000			\hrulefill
				\item 	국경없는 의사회	\hrulefill 20일 CMS \\ 					20,000			\hrulefill
				\item 	쿠쿠홈시스		\hrulefill 	일 \\ 						9,900			\hrulefill
				\item 	넷플렉스			\hrulefill 20일 \\ 						14,500			\hrulefill

				\item 
			\end{itemize}
			\end{block}						
		\end{frame}						


	%	---------------------------------------------------------- page	4
		\begin{frame} [t,plain]
		\frametitle{신용카드}
			\begin{block} {신용카드}
			\setlength{\leftmargini}{1em}			
			\begin{itemize}
				\item 	카드			\hrulefill 롯데 \\ 						
											4061 5800 0287 4211		\hrulefill \\
											20일 	\hrulefill \\
											신한은행 	\hrulefill

				\item 	체크카드		\hrulefill 카카오 \\ 								
											5365 1002 5975 2307	\hrulefill

				\item 	체크카드		\hrulefill 우체국 \\ 									\hrulefill

				\item 	동구페이		\hrulefill 코나카드 \\ 						
											9465 4442 0002 4700	\hrulefill

			\end{itemize}
			\end{block}						
		\end{frame}						



	%	---------------------------------------------------------- 주요 날짜 }
	%		Frame
	%	----------------------------------------------------------
		\section{주요 날짜 }

	%	---------------------------------------------------------- 1		page
		\begin{frame} [t,plain]
		\frametitle{주요 날짜 }
			\begin{block} {주요 날짜 }
			\setlength{\leftmargini}{2em}			
			\begin{itemize}
				\item 월급날 :\hrulefill 30일
			\end{itemize}
			\end{block}						
		\end{frame}						


	%	---------------------------------------------------------- 2 	page
		\begin{frame} [t,plain]
			\begin{block} {제사 }
			\setlength{\leftmargini}{2em}			
			\begin{itemize}
				\item  할아버지 제사 : 
				\item  아버지 제사 : 5월 7일 
				\item 장인 제사 : 
				\item 현찬섭 제사 : 
			\end{itemize}
			\end{block}						
		\end{frame}						


	%	---------------------------------------------------------- 3		page
		\begin{frame} [t,plain]
			\begin{block} {생일 }
			\setlength{\leftmargini}{2em}			
			\begin{itemize}
				\item  안신영
				\item 김재민
				\item 김재경
				\item 김재홍
			\end{itemize}
			\end{block}						

		\end{frame}						


	%	---------------------------------------------------------- 4		page
		\begin{frame} [t,plain]
		\end{frame}						


	%	---------------------------------------------------------- 영화 }
	%		Frame
	%	----------------------------------------------------------
		\section{영화 }

	%	---------------------------------------------------------- 1	page
		\begin{frame} [t,plain]
		\frametitle 	{영화}

			\begin{block} {영화 - 한국 }
			\setlength{\leftmargini}{2em}			
			\begin{itemize}
				\item 
				\item 
				\item 
			\end{itemize}
			\end{block}						
		\end{frame}						

	%	---------------------------------------------------------- 2	page
		\begin{frame} [t,plain]
			\begin{block} {영화 - 영미권}
			\setlength{\leftmargini}{2em}			
			\begin{itemize}
				\item 
				\item 
				\item 
			\end{itemize}
			\end{block}						
		\end{frame}						


	%	---------------------------------------------------------- 3	page
		\begin{frame} [t,plain]
			\begin{block} {영화 - 중국}
			\setlength{\leftmargini}{2em}			
			\begin{itemize}
				\item 
				\item 
				\item 
			\end{itemize}
			\end{block}						
		\end{frame}						

	%	---------------------------------------------------------- 4	page
		\begin{frame} [t,plain]
			\begin{block} {영화 - 인도}
			\setlength{\leftmargini}{2em}			
			\begin{itemize}
				\item 
				\item 
				\item 
			\end{itemize}
			\end{block}						
		\end{frame}						



%	%	---------------------------------------------------------- 수채화 }
%	%		Frame
%	%	----------------------------------------------------------
%		\section{중앙도서관 수채화 수업}
%
%		\begin{frame} [t,plain]
%		\frametitle{중앙도서관 수채화 수업}
%			\begin{block} {중앙도서관 수채화 수업}
%			\setlength{\leftmargini}{2em}			
%			\begin{itemize}
%				\item 	이종원
%				\item 
%			\end{itemize}
%			\end{block}						
%		\end{frame}						
%
%
%		\begin{frame} [t,plain]
%		\frametitle{중앙도서관 수채화 수업}
%			\begin{block} {중앙도서관 수채화 수업}
%			\setlength{\leftmargini}{2em}			
%			\begin{itemize}
%				\item 	07/04	
%				\item 	07/11	
%				\item 	07/18
%				\item 	08/01	
%				\item 	08/08	
%				\item 	08/1	5
%				\item 	08/22
%				\item 	08/29	
%				\item 	09/05	
%				\item 	09/12	
%				\item 	09/19	
%				\item 	09/26	
%
%				\item 
%			\end{itemize}
%			\end{block}						
%		\end{frame}						
%
%		\begin{frame} [t,plain]
%		\frametitle{중앙도서관 수채화 수업}
%			\begin{block} {중앙도서관 수채화 수업 : 10 월}
%			\setlength{\leftmargini}{2em}			
%			\begin{itemize}
%
%				\item 	10/03	
%				\item 	10/10	
%				\item 	10/17	
%				\item 	10/24 억새밭 그리기 \hrulefill \\
%						포교사 수계  결석
%				\item 	10/28
%				\item 
%			\end{itemize}
%			\end{block}						
%		\end{frame}						
%
%
%		\begin{frame} [t,plain]
%		\frametitle{중앙도서관 수채화 수업}
%			\begin{block} {중앙도서관 수채화 수업 : 11월 }
%			\setlength{\leftmargini}{2em}			
%			\begin{itemize}
%
%				\item 	11/07	
%				\item 	11/14	
%				\item 	11/21	
%				\item 	11/28	
%				\item 	12/05	
%	
%
%				\item 
%			\end{itemize}
%			\end{block}						
%		\end{frame}						
%
%

	%	---------------------------------------------------------- 하드디스크 }
	%		Frame
	%	----------------------------------------------------------
		\section{하드디스크 }

	%	---------------------------------------------------------- 1	page
		\begin{frame} [t,plain]
		\frametitle 	{하드디스크}
		\end{frame}						

	%	---------------------------------------------------------- 2	page
		\begin{frame} [t,plain]
			\begin{block} {하드디스크 }
			\setlength{\leftmargini}{2em}			
			\begin{itemize}
				\item 3.5인치 HDD
				\item 2.5인치 HDD
				\item 2.5인치 SDD
				\item USB메모리
			\end{itemize}
			\end{block}						
		\end{frame}						


	%	---------------------------------------------------------- 3		page
		\begin{frame} [t,plain]

			\begin{block} {하드디스크 3.5인치 HDD }
			\setlength{\leftmargini}{2em}			
			\begin{itemize}
				\item 1 \hrulefill 
				\item 2	\hrulefill 
				\item 3	\hrulefill 
				\item 4	\hrulefill 
				\item 5 	\hrulefill 
				\item 6	\hrulefill 
				\item 7	\hrulefill 
				\item 8	\hrulefill 
				\item 9  \hrulefill 
				\item 10		\hrulefill 
				\item 11		\hrulefill 
				\item 12		\hrulefill 

			\end{itemize}
			\end{block}						

		\end{frame}						


	%	---------------------------------------------------------- 4		page
		\begin{frame} [t,plain]

			\begin{block} {하드디스크 2.5인치 HDD }
			\setlength{\leftmargini}{2em}			
			\begin{itemize}
				\item 1 작업용 		\hrulefill 사무실
				\item 2	보고서		\hrulefill  사무실
				\item 3	메인 백업	\hrulefill 	 사무실 
				\item 4
			\end{itemize}
			\end{block}			
			
		\end{frame}						



	%	---------------------------------------------------------- 5		page
		\begin{frame} [t,plain]

			\begin{block} {하드디스크 2.5인치 SDD }
			\setlength{\leftmargini}{2em}			
			\begin{itemize}
				\item 1 
				\item 2	
				\item 3	
				\item 4
			\end{itemize}
			\end{block}						
			
		\end{frame}						

	%	---------------------------------------------------------- 6		page 	USB 메모리 
		\begin{frame} [t,plain]


			\begin{block}  {USB 메모리 }
			\setlength{\leftmargini}{2em}			
			\begin{itemize}

				\item 들고 다님
				\item [01]		31.9 GB 중 6.25 GB 사용가능
				\item [04] 	7.43 GB 작업용 
				\item [이동용작업] 7.45 GB 중 467 MB 사용가능 \\.

				\item 외장 02케이스에 넣어 2층 책장에
				\item [02]	
				\item [03]	14.9 GB 중 5.12 GB 사용가능
				\item [05]	3.76 GB 음악용 \\.

				\item 2층 벽에 걸어둠
				\item [우체국] 29.8 GB 중 27.6 GB 사용가능 \\공인인증서용으로 사용
				\item [설치용] 14.9 GB 중 1.95 GB 사용가능 \\시스템 설치용

			\end{itemize}
			\end{block}						

		\end{frame}						

	%	---------------------------------------------------------- 7	page		  설치 프로그램
		\begin{frame} [t,plain]

			\begin{block}  {설치 프로그램 : 기본 }
			\setlength{\leftmargini}{2em}			
			\begin{itemize}
				\item [01] 알집
				\item [01] 카카오톡
				\item [05] 	오피스
				\item [05]		아파치
				\item [05] 	폴라리스
				\item [05] 	ADOBE
			\end{itemize}
			\end{block}						



		\end{frame}						


	%	---------------------------------------------------------- 8	page
		\begin{frame} [t,plain]

			\begin{block}  {설치 프로그램 : Git }
			\setlength{\leftmargini}{2em}			
			\begin{itemize}
				\item [12]	 Git
				\item [12]	 GitHub
				\item [12]	 npp
				\item [16]		소스트리
				\item [18]		VS Code
			\end{itemize}
			\end{block}						

			\begin{block}  {설치 프로그램 : }
			\setlength{\leftmargini}{2em}			
			\begin{itemize}
				\item [21] 	Texlive 2020
				\item [21] 	Autocad
				\item [21] 	Mathcad
				\item [21] 	Adobe
			\end{itemize}
			\end{block}						

		\end{frame}						

	%	---------------------------------------------------------- 회원 가입 }
	%		Frame
	%	----------------------------------------------------------
		\section{회원 가입 }

	%	---------------------------------------------------------- 	page	1
		\begin{frame} [t,plain]
		\frametitle{회원 가입 }
		\end{frame}						

	%	---------------------------------------------------------- 	page	2
		\begin{frame} [t,plain]

			\begin{block} {회원 가입: 전자책 도서관 }
			\setlength{\leftmargini}{2em}			
			\begin{itemize}
				\item 강서 \\
						h 010 3839 5609 \hrulefill \\
						h . 789 456 123 \hrulefill \\
				\item 
				\item 
			\end{itemize}
			\end{block}						

		\end{frame}						



	%	---------------------------------------------------------- 	page	3
		\begin{frame} [t,plain]

			\begin{block} {회원 가입 : 오피스 365}
			\setlength{\leftmargini}{2em}			
			\begin{itemize}
				\item 	김재경 \\
						173492 @ donga.ac.kr \hrulefill \\
						Aalice 10308 ! \hrulefill \\
				\item 
				\item 
			\end{itemize}
			\end{block}						

		\end{frame}						

	%	---------------------------------------------------------- 	page	4
		\begin{frame} [t,plain]
		\end{frame}						





	%	---------------------------------------------------------- 클라우드 
	%		Frame
	%	----------------------------------------------------------
		\section{클라우드 }

	%	---------------------------------------------------------- 	page	1
		\begin{frame} [t,plain]
		\frametitle{ 클라우드 }
		\end{frame}						

	%	---------------------------------------------------------- 	page	2
		\begin{frame} [t,plain]
			\begin{block} {클라우드}
			\setlength{\leftmargini}{2em}			
			\begin{itemize}
				\item a 개인				\hfill 	15 / 02.0
				\item b 정산서,프로그램	\hfill 	15 / 03.4
				\item c 책 스캔			\hfill 	15 /	12.9
				\item d 성인				\hfill 	15 / 14.6
				\item e 올재				\hfill 	15 / 00.2
				\item f 포교사			\hfill 	15 / 00.0
				\item g 					\hfill 	15 / 00.0

				\item h 메인 작업용 		\hfill 	15 / 14.8
				\item i 					\hfill 	15 / 00.0
				\item j 					\hfill 	15 / 00.0
				\item k 					\hfill 	15 / 00.0

				\item L Latex 			\hfill 	15 / 12.2
				\item m 여행사진 		\hfill 	15 / 14.7
				\item n 음악 			\hfill 	15 / 02.2
				\item o 강좌				\hfill 	15 / 11.6
				\item p 불교				\hfill 	15 / 00.0
				\item q 만해아카데미		\hfill 	15 / 05.4 \\ 
				\hrulefill

				\item naver				\hfill 	15 / 00.2	
			\end{itemize}
			\end{block}						


		\end{frame}						

	%	---------------------------------------------------------- 위생 }
	%		Frame
	%	---------------------------------------------------------- 	page	3
		\section{위생 }
		\begin{frame} [t,plain]
		\frametitle{위생 }
			\begin{block} {위생 }
			\setlength{\leftmargini}{2em}			
			\begin{itemize}
				\item 09-20 이발 
				\item 10-28 이발 
				\item 11-21 이발 
				\item 12-19 이발  쥬댐 
				\item 
				\item 
			\end{itemize}
			\end{block}						
		\end{frame}						

	%	---------------------------------------------------------- 	page	4
		\begin{frame} [t,plain]
		\end{frame}						



	%	========================================================== 물품 구입
	%
	%
	%
	%	---------------------------------------------------------- 	page	1

		\part{물품 구입}
		\frame{\partpage}
		
\label{part5} 	%  물품 구입

	%	---------------------------------------------------------- 	page	2
		\begin{frame} [plain]{목차}
		\tableofcontents%
		\end{frame}

	%	---------------------------------------------------------- 만년필
	%		Frame
	%	---------------------------------------------------------- 	page	3
		\section{만년필}

		\begin{frame} [t,plain]
		\frametitle{만년필}

			\begin{block} {촉의 구분}
			\setlength{\leftmargini}{2em}			
			\begin{itemize}
				\item SEf EF F M B BB OM OB OBB
			\end{itemize}
			\end{block}						


			\begin{block} {라미}
			\setlength{\leftmargini}{2em}			
			\begin{itemize}
				\item 삷을 지혜롭게
				\item 마음을 자비롭게
				\item 세상을 자유롭게
			\end{itemize}
			\end{block}						

			\begin{block} {세일러 레큘레}
			\setlength{\leftmargini}{2em}			
			\begin{itemize}
				\item 일본의 만년필 제조회사. 파이롯트, 플래티넘과 함께 일본 3대 만년필 제조사로 불린다.
				\item 마음을 자비롭게
				\item 세상을 자유롭게
			\end{itemize}
			\end{block}						

		\end{frame}						


	%	---------------------------------------------------------- 만년필 라미
	%		Frame
	%	---------------------------------------------------------- 	page	4
		\section{만년필 : 라미 }

		\begin{frame} [t,plain]
		\frametitle{만년필 : 라미}

			\begin{block} {라미}
			\setlength{\leftmargini}{2em}			
			\begin{itemize}
				\item 삷을 지혜롭게
				\item 마음을 자비롭게
				\item 세상을 자유롭게
			\end{itemize}
			\end{block}						

		\end{frame}						


	%	---------------------------------------------------------- 만년필 : 세일러
	%		Frame
	%	---------------------------------------------------------- 	page	5
		\section{만년필 : 세일러 }

		\begin{frame} [t,plain]
		\frametitle{만년필 : 세일러 }

			\begin{block} {세일러 레큘레}
			\setlength{\leftmargini}{2em}			
			\begin{itemize}
				\item 일본의 만년필 제조회사. 파이롯트, 플래티넘과 함께 일본 3대 만년필 제조사로 불린다.
				\item 마음을 자비롭게
				\item 세상을 자유롭게
			\end{itemize}
			\end{block}						

		\end{frame}						


	%	---------------------------------------------------------- 만년필 : 다이소
	%		Frame
	%	---------------------------------------------------------- 	page	6
		\section{만년필 : 다이소 }

		\begin{frame} [t,plain]
		\frametitle{만년필 : 다이소 }

			\begin{block} {다이소}
			\setlength{\leftmargini}{2em}			
			\begin{itemize}
				\item 일본의 만년필 제조회사. 파이롯트, 플래티넘과 함께 일본 3대 만년필 제조사로 불린다.
				\item 마음을 자비롭게
				\item 세상을 자유롭게
			\end{itemize}
			\end{block}						

		\end{frame}						

	%	---------------------------------------------------------- 만년필 : 모나미
	%		Frame
	%	---------------------------------------------------------- 	page	7
		\section{만년필 : 모나미 }

		\begin{frame} [t,plain]
		\frametitle{만년필 : 모나미 }

			\begin{block} {모나미}
			\setlength{\leftmargini}{2em}			
			\begin{itemize}
				\item 일본의 만년필 제조회사. 파이롯트, 플래티넘과 함께 일본 3대 만년필 제조사로 불린다.
				\item 마음을 자비롭게
				\item 세상을 자유롭게
			\end{itemize}
			\end{block}						

		\end{frame}						

	%	---------------------------------------------------------- 	page	8
		\begin{frame} [t,plain]
		\end{frame}						



	%	========================================================== 좋은 문구
	%
	%
	%
	%	---------------------------------------------------------- 	page	1

		\part{좋은 문구}
		\frame{\partpage}
		
\label{part4} 	%  좋은 문구

	%	---------------------------------------------------------- 	page	2
		\begin{frame} [plain]{목차}
		\tableofcontents%
		\end{frame}

	%	---------------------------------------------------------- 	page	3
		\begin{frame} [t,plain]
		\end{frame}						
	%	---------------------------------------------------------- 	page	4
		\begin{frame} [t,plain]
		\end{frame}						



	%	---------------------------------------------------------- 좋은 문구}
	%		Frame
	%	---------------------------------------------------------- 	page	1
		\section{좋은 문구}

		\begin{frame} [t,plain]
		\frametitle{좋은 문구}
			\begin{block} {좋은 문구}
			\setlength{\leftmargini}{2em}			
			\begin{itemize}
				\item 삷을 지혜롭게
				\item 마음을 자비롭게
				\item 세상을 자유롭게

			\end{itemize}
			\end{block}						
		\end{frame}						


	%	---------------------------------------------------------- 	page	2
		\begin{frame} [t,plain]
		\end{frame}						
	%	---------------------------------------------------------- 	page	3
		\begin{frame} [t,plain]
		\end{frame}						
	%	---------------------------------------------------------- 	page	4
		\begin{frame} [t,plain]
		\end{frame}						


% ------------------------------------------------------------------------------
% End document
% ------------------------------------------------------------------------------





\end{document}


	%	---------------------------------------------------------- 	page	1
		\begin{frame} [t,plain]
		\end{frame}						

