
%	-------------------------------------------------------------------------------
%			물품 구입 정리
%
%			작성		
%				2021년 
%				1월 
%				8일 
%				월
%
%
%
%
%
%
%	-------------------------------------------------------------------------------

%	\documentclass[10pt,xcolor=pdftex,dvipsnames,table]{beamer}
%	\documentclass[10pt,blue,xcolor=pdftex,dvipsnames,table,handout]{beamer}
%	\documentclass[14pt,blue,xcolor=pdftex,dvipsnames,table,handout]{beamer}
	\documentclass[aspectratio=1610,20pt,xcolor=pdftex,dvipsnames,table,handout]{beamer}
%	\documentclass[aspectratio=169,17pt,xcolor=pdftex,dvipsnames,table,handout]{beamer}
%	\documentclass[aspectratio=149,17pt,xcolor=pdftex,dvipsnames,table,handout]{beamer}
%	\documentclass[aspectratio=54,17pt,xcolor=pdftex,dvipsnames,table,handout]{beamer}
%	\documentclass[aspectratio=43,17pt,xcolor=pdftex,dvipsnames,table,handout]{beamer}
%	\documentclass[aspectratio=32,17pt,xcolor=pdftex,dvipsnames,table,handout]{beamer}

		% Font Size
		%	default font size : 11 pt
		%	8,9,10,11,12,14,17,20
		%
		% 	put frame titles 
		% 		1) 	slideatop
		%		2) 	slide centered
		%
		%	navigation bar
		% 		1)	compress
		%		2)	uncompressed
		%
		%	Color
		%		1) blue
		%		2) red
		%		3) brown
		%		4) black and white	
		%
		%	Output
		%		1)  	[default]	
		%		2)	[handout]		for PDF handouts
		%		3) 	[trans]		for PDF transparency
		%		4)	[notes=hide/show/only]

		%	Text and Math Font
		% 		1)	[sans]
		% 		2)	[sefif]
		%		3) 	[mathsans]
		%		4)	[mathserif]


		%	---------------------------------------------------------	
		%	슬라이드 크기 설정 ( 128mm X 96mm )
		%	---------------------------------------------------------	
%			\setbeamersize{text margin left=2mm}
%			\setbeamersize{text margin right=2mm}

		%	---------------------------------------------------------	
		%	슬라이드 크기 설정 ( 128mm X 96mm )
		%	---------------------------------------------------------	

%			% Format presentation size to A4
%			\usepackage[size=a4]{beamerposter}		% A4용지 크기 사용
			\geometry{paper=a5paper}
%			% Format presentation size to A4 길게
%			\geometry{paper=a4paper, landscape}

			\setbeamersize{text margin left=10mm}
			\setbeamersize{text margin right=10mm}


	%	========================================================== 	Package
		\usepackage{kotex}						% 한글 사용
		\usepackage{amssymb,amsfonts,amsmath}	% 수학 수식 사용
		\usepackage{color}					%
		\usepackage{colortbl}					%

		\usepackage{caption}		% 테이블 사용 건 2020.10.22
		\usepackage{tabularx} 	% 테이블 사용 건 2020.10.22


	%		========================================================= 	note 옵션인 
	%			\setbeameroption{show only notes}
		

	%		========================================================= 	Theme

		%	---------------------------------------------------------	
		%	전체 테마
		%	---------------------------------------------------------	
		%	테마 명명의 관례 : 도시 이름
%			\usetheme{default}			%
%			\usetheme{Madrid}    		%
%			\usetheme{CambridgeUS}    	% -red, no navigation bar
%			\usetheme{Antibes}			% -blueish, tree-like navigation bar

		%	----------------- table of contents in sidebar
			\usetheme{Berkeley}		% -blueish, table of contents in sidebar
									% 개인적으로 마음에 듬

%			\usetheme{Marburg}			% - sidebar on the right
%			\usetheme{Hannover}		% 왼쪽에 마크
%			\usetheme{Berlin}			% - navigation bar in the headline
%			\usetheme{Szeged}			% - navigation bar in the headline, horizontal lines
%			\usetheme{Malmoe}			% - section/subsection in the headline

%			\usetheme{Singapore}
%			\usetheme{Amsterdam}

		%	---------------------------------------------------------	
		%	색 테마
		%	---------------------------------------------------------	
%			\usecolortheme{albatross}	% 바탕 파란
%			\usecolortheme{crane}		% 바탕 흰색
%			\usecolortheme{beetle}		% 바탕 회색
%			\usecolortheme{dove}		% 전체적으로 흰색
%			\usecolortheme{fly}		% 전체적으로 회색
%			\usecolortheme{seagull}	% 휜색
%			\usecolortheme{wolverine}	& 제목이 노란색
%			\usecolortheme{beaver}

		%	---------------------------------------------------------	
		%	Inner Color Theme 			내부 색 테마 ( 블록의 색 )
		%	---------------------------------------------------------	

%			\usecolortheme{rose}		% 흰색
%			\usecolortheme{lily}		% 색 안 칠한다
%			\usecolortheme{orchid} 	% 진하게

		%	---------------------------------------------------------	
		%	Outter Color Theme 		외부 색 테마 ( 머리말, 고리말, 사이드바 )
		%	---------------------------------------------------------	

%			\usecolortheme{whale}		% 진하다
%			\usecolortheme{dolphin}	% 중간
%			\usecolortheme{seahorse}	% 연하다

		%	---------------------------------------------------------	
		%	Font Theme 				폰트 테마
		%	---------------------------------------------------------	
%			\usfonttheme{default}		
			\usefonttheme{serif}			
%			\usefonttheme{structurebold}			
%			\usefonttheme{structureitalicserif}			
%			\usefonttheme{structuresmallcapsserif}			



		%	---------------------------------------------------------	
		%	Inner Theme 				
		%	---------------------------------------------------------	

%			\useinnertheme{default}
			\useinnertheme{circles}		% 원문자			
%			\useinnertheme{rectangles}		% 사각문자			
%			\useinnertheme{rounded}			% 깨어짐
%			\useinnertheme{inmargin}			




		%	---------------------------------------------------------	
		%	이동 단추 삭제
		%	---------------------------------------------------------	
%			\setbeamertemplate{navigation symbols}{}

		%	---------------------------------------------------------	
		%	문서 정보 표시 꼬리말 적용
		%	---------------------------------------------------------	
%			\useoutertheme{infolines}


			
	%	---------------------------------------------------------- 	배경이미지 지정
%			\pgfdeclareimage[width=\paperwidth,height=\paperheight]{bgimage}{./fig/Chrysanthemum.jpg}
%			\setbeamertemplate{background canvas}{\pgfuseimage{bgimage}}

		%	---------------------------------------------------------	
		% 	본문 글꼴색 지정
		%	---------------------------------------------------------	
%			\setbeamercolor{normal text}{fg=purple}
%			\setbeamercolor{normal text}{fg=red!80}	% 숫자는 투명도 표시


		%	---------------------------------------------------------	
		%	itemize 모양 설정
		%	---------------------------------------------------------	
%			\setbeamertemplate{items}[ball]
%			\setbeamertemplate{items}[circle]
%			\setbeamertemplate{items}[rectangle]






		\setbeamercovered{dynamic}





		% --------------------------------- 	문서 기본 사항 설정
		\setcounter{secnumdepth}{3} 		% 문단 번호 깊이
		\setcounter{tocdepth}{3} 			% 문단 번호 깊이




% ------------------------------------------------------------------------------
% Begin document (Content goes below)
% ------------------------------------------------------------------------------
	\begin{document}
	

			\title 	{물품구입}
			\author 	{ 김대희 }
			\date 	{ 
					2021년 
					1월 
					8일
					금  
					} 


% -----------------------------------------------------------------------------
%		개정 내용
% -----------------------------------------------------------------------------
%
%		2021년 1월 8일 첫제작
%
%
%
%


	%	==========================================================
	%
	%	---------------------------------------------------------- 1	page

		\begin{frame}[plain]
		\titlepage
		\end{frame}



	%	========================================================== 물품 구입
	%
	%
	%
	%	---------------------------------------------------------- 	page	1

		\part{만년필}
		\frame{\partpage}
		
	%	---------------------------------------------------------- 	page	2
		\begin{frame} [plain]{목차}
		\tableofcontents%
		\end{frame}

	%	---------------------------------------------------------- 만년필
	%		Frame
	%	---------------------------------------------------------- 	page	3
		\section{만년필}

		\begin{frame} [t,plain]
		\frametitle{만년필}

			\begin{block} {촉의 구분}
			\setlength{\leftmargini}{2em}			
			\begin{itemize}
				\item SEf EF F M B BB OM OB OBB
			\end{itemize}
			\end{block}						


			\begin{block} {라미}
			\setlength{\leftmargini}{2em}			
			\begin{itemize}
				\item 삷을 지혜롭게
				\item 마음을 자비롭게
				\item 세상을 자유롭게
			\end{itemize}
			\end{block}						

			\begin{block} {세일러 레큘레}
			\setlength{\leftmargini}{2em}			
			\begin{itemize}
				\item 일본의 만년필 제조회사. 파이롯트, 플래티넘과 함께 일본 3대 만년필 제조사로 불린다.
				\item 마음을 자비롭게
				\item 세상을 자유롭게
			\end{itemize}
			\end{block}						

		\end{frame}						


	%	---------------------------------------------------------- 만년필 라미
	%		Frame
	%	---------------------------------------------------------- 	page	4
		\section{만년필 : 라미 }

		\begin{frame} [t,plain]
		\frametitle{만년필 : 라미}

			\begin{block} {라미}
			\setlength{\leftmargini}{2em}			
			\begin{itemize}
				\item 삷을 지혜롭게
				\item 마음을 자비롭게
				\item 세상을 자유롭게
			\end{itemize}
			\end{block}						

		\end{frame}						


	%	---------------------------------------------------------- 만년필 : 세일러
	%		Frame
	%	---------------------------------------------------------- 	page	5
		\section{만년필 : 세일러 }

		\begin{frame} [t,plain]
		\frametitle{만년필 : 세일러 }

			\begin{block} {세일러 레큘레}
			\setlength{\leftmargini}{2em}			
			\begin{itemize}
				\item 일본의 만년필 제조회사. 파이롯트, 플래티넘과 함께 일본 3대 만년필 제조사로 불린다.
				\item 마음을 자비롭게
				\item 세상을 자유롭게
			\end{itemize}
			\end{block}						

		\end{frame}						


	%	---------------------------------------------------------- 만년필 : 다이소
	%		Frame
	%	---------------------------------------------------------- 	page	6
		\section{만년필 : 다이소 }

		\begin{frame} [t,plain]
		\frametitle{만년필 : 다이소 }

			\begin{block} {다이소}
			\setlength{\leftmargini}{2em}			
			\begin{itemize}
				\item 일본의 만년필 제조회사. 파이롯트, 플래티넘과 함께 일본 3대 만년필 제조사로 불린다.
				\item 마음을 자비롭게
				\item 세상을 자유롭게
			\end{itemize}
			\end{block}						

		\end{frame}						

	%	---------------------------------------------------------- 만년필 : 트위스비
	%		Frame
	%	---------------------------------------------------------- 	page	7
		\section{만년필 : 트위스비 }

		\begin{frame} [t,plain]
		\frametitle{만년필 : 트위스비 }

			\begin{block} {트위스비}
			\setlength{\leftmargini}{2em}			
			\begin{itemize}
				\item 다이아몬드 580 AL R 
				\item Vac 700 R
				\item 
			\end{itemize}
			\end{block}						

		\end{frame}						

	%	---------------------------------------------------------- 만년필 : 모나미
	%		Frame
	%	---------------------------------------------------------- 	page	8
		\section{만년필 : 모나미 }

		\begin{frame} [t,plain]
		\frametitle{만년필 : 모나미 }

			\begin{block} {모나미}
			\setlength{\leftmargini}{2em}			
			\begin{itemize}
				\item 일본의 만년필 제조회사. 파이롯트, 플래티넘과 함께 일본 3대 만년필 제조사로 불린다.
				\item 마음을 자비롭게
				\item 세상을 자유롭게
			\end{itemize}
			\end{block}						

		\end{frame}						



	%	========================================================== 좋은 문구
	%
	%
	%
	%	---------------------------------------------------------- 	page	1

		\part{좋은 문구}
		\frame{\partpage}
		
\label{part4} 	%  좋은 문구

	%	---------------------------------------------------------- 	page	2
		\begin{frame} [plain]{목차}
		\tableofcontents%
		\end{frame}

	%	---------------------------------------------------------- 	page	3
		\begin{frame} [t,plain]
		\end{frame}						
	%	---------------------------------------------------------- 	page	4
		\begin{frame} [t,plain]
		\end{frame}						



	%	---------------------------------------------------------- 좋은 문구}
	%		Frame
	%	---------------------------------------------------------- 	page	1
		\section{좋은 문구}

		\begin{frame} [t,plain]
		\frametitle{좋은 문구}
			\begin{block} {좋은 문구}
			\setlength{\leftmargini}{2em}			
			\begin{itemize}
				\item 삷을 지혜롭게
				\item 마음을 자비롭게
				\item 세상을 자유롭게

			\end{itemize}
			\end{block}						
		\end{frame}						


	%	---------------------------------------------------------- 	page	2
		\begin{frame} [t,plain]
		\end{frame}						
	%	---------------------------------------------------------- 	page	3
		\begin{frame} [t,plain]
		\end{frame}						
	%	---------------------------------------------------------- 	page	4
		\begin{frame} [t,plain]
		\end{frame}						


% ------------------------------------------------------------------------------
% End document
% ------------------------------------------------------------------------------





\end{document}


	%	---------------------------------------------------------- 	page	1
		\begin{frame} [t,plain]
		\end{frame}						

